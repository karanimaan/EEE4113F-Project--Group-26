% ----------------------------------------------------
% Literature Review
% ----------------------------------------------------
\documentclass[class=report,11pt,crop=false]{standalone}
\input{../Style/ChapterStyle.tex}

\begin{document}
\ifstandalone
\tableofcontents
\fi
% ----------------------------------------------------
\chapter{Literature Review \label{ch:literature}}
\vspace{0.5cm}
% ----------------------------------------------------
\section{Current Weighing Methods}
This section explores and evaluates the different weighing methods used in ornithological research today with a focus on smaller birds (those weighing less than 100g).

\subsection{Spring Scales}
Spring scales measure weight based on the extension of a spring when a force (the weight of the bird) is applied. Their main advantage, as described by Manolis \cite{manoils2024simple}, is that they are “relatively inexpensive and sufficiently portable to suffice for short-term field project[s]”. However, within the same study, the scale was only accurate to within 0,5g and when smaller birds can weigh less than 50g, spring scales may lack the precision for such research applications. Readings can also be influenced by factors such as calibration drift.

\subsection{Electronic Scales}
Electronic scales utilize load cells or strain gauges to convert the weight of the bird into an electronic signal, which can then be displayed on a digital screen. These scales offer precise measurements and are commonly used in both laboratory and field settings. This is shown in Carpenter et al. \cite{carpenter1983weight} paper where they were able to improve the precision of their measurements from 0,05g to 0,01g by simply replacing their spring scales with electronic ones. Another advantage over spring scales is that they do not have to be recalibrated after moving \cite{carpenter1983weight} and they tend to come with features such as taring functions to account for container weight or bird movements.

\subsection{Perching Scales}
Perching scales integrate a weighing platform into an artificial perch or nest. In Poole and Shoukimas’ \cite{poole1982scale} study, birds landing on perch would deflect a transducer (a metal beam with 4 strain gauges bonded to it), thus generating an electronic signal. Reid et al. \cite{reid1999measurement} used artificial nests rigged with a load cell in much the same way. In both studies, these electronic signals would then be recorded via some kind of electronic storage medium. This meant the birds could be weighed remotely, which minimizes stress and reduces the risk of injury, making perching scales particularly useful for long-term monitoring studies or behavioural observations. However, for such long-term studies, researchers would need to keep track of a large number of birds, which would also result in a large amount of data that needs to be stored.

Manolis \cite{manoils2024simple} provides a solution to these issues by urging other researchers to make use of telemetric equipment. One such technology is Radio Frequency Identification (RFID) which enables researchers to track individual birds and record their weight automatically. Wang et al. \cite{wang2019rfid} made use of RFID by attaching two transponders to each bird, which would be detected by antennas placed under the perches. When a tagged bird interacts with the RFID reader, its unique identifier and weight are recorded electronically. This makes the data much easier to organise. It also reduces the volume of data as the weight is only taken when the birds are on top of the perch, allowing researchers to collect data on a larger scale. 


\section{Data Transmission and User Interface}
The client requires the weight data to be retrieved remotely. We need to figure out two things: where do we want the data to be sent, and how do we want it to be sent.
This section investigates different options for transmitting data from a microprocessor. In a comparative performance study by Eridani et al., three protocols were compared: Bluetooth, Wi-Fi direct, and ESP-NOW ("a new protocol that allows multiple devices to communicate with each other without the use of Wi-Fi, with low power consumption" \cite{comparitiveEspnow}). 5 metrics were used in the tests: maximum range, transmission speed, latency, power usage, and signal resistance to obstructions \cite{comparitiveEspnow}. A brief summary of the performance of each protocol is shown below in Figure \ref{fig:performance}.
\begin{figure}[h]
	\centering
	\includegraphics[width=0.7\linewidth]{Figures/performance}
	\caption{Overall Performance of each Protocol 			\cite{comparitiveEspnow}}
	\label{fig:performance}
\end{figure}

ESP-NOW performs best in range and latency; Bluetooth in power usage; and Wi-Fi in transmission speed. In our context, power usage would be most important. Bluetooth seems to provide sufficient range and speed.

The problem with this; however, is that connecting the system to the user's phone requires effort and perhaps expertise that the user may not have. In this case, connecting the system to the internet may be a better option (if internet connection is available, i.e. if eduroam is in range). 

Budoyo and Andriana used the internet when designing a prototype of a digital scale to measure the weight of onions. \cite{iot}. They interfaced the microcontroller (an  ATMega2560) to the internet using an ESP8266 Wi-Fi module. The weight data is sent to a website where it is stored in a database.

\section{Challenges and Considerations}
While modern weighing methods offer significant advantages in terms of accuracy, convenience, and animal welfare, researchers must consider several factors when selecting the most appropriate technique for their study.

\subsection{Size and Species}
The size and behaviour of the target bird species may influence the suitability of different weighing methods. Some birds may become skittish around researchers which would make measurements unreliable. In Manolis’ case \cite{manoils2024simple}, they had to use binoculars to take readings of the scale from afar; an inconvenience that is entirely removed from the solutions presented by Poole & Shoukimas \cite{poole1982scale} and Reid et al. \cite{reid1999measurement}. Smaller birds may require scales with higher precision, while larger species may benefit from perching scale systems capable of handling multiple individuals simultaneously.

\subsection{Environmental Conditions}
Field studies often expose equipment to challenging environmental conditions. For example, Manolis \cite{manoils2024simple} had to keep swaying to a minimum to get accurate readings, hence spring scales would not be suitable in windy conditions. Rain can seriously damage electronic components so it was important for Reid et al. \cite{reid1999measurement} to house the amplifier unit in “a small water-resistant case with a sealed connection to the data logger cable”. Researchers must choose weighing methods that are robust and reliable under these circumstances, with weather-resistant features where necessary.

\subsection{Ethical Considerations}
Ethical guidelines emphasize the importance of minimizing stress and harm to the animals being studied. This does not necessarily mean all researchers must prioritize the use of non-invasive weighing techniques such as perch scale, as these have their considerations. The impact of RFID tags on the birds like those used by Wang et al. \cite{wang2019rfid} must be taken into account. There are also methods of making more traditional weighing techniques less invasive, such as Manolis \cite{reid1999measurement} using sunflower seeds to entice the birds to land on the scale. No one weighing technique is superior to another in this regard. It is care that researchers implore in their methodology that will decide if the birds make it to the end of the study unharmed.

% ----------------------------------------------------
\ifstandalone
\bibliography{../Bibliography/References.bib}
\printnoidxglossary[type=\acronymtype,nonumberlist]
\fi
\end{document}
% ----------------------------------------------------