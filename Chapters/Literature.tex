% ----------------------------------------------------
% Literature Review
% ----------------------------------------------------
\documentclass[class=report,11pt,crop=false]{standalone}
\input{../Style/ChapterStyle.tex}
\input{../FrontMatter/Glossary.tex}

\begin{document}
\ifstandalone
\tableofcontents
\fi
% ----------------------------------------------------
\chapter{Literature Review \label{ch:literature}}
\epigraph{If you wish to make an apple pie from scratch, you must first invent the universe.}%
    {\emph{---Carl Sagan}}
\vspace{0.5cm}
% ----------------------------------------------------

\section{Data Transmission and User Interface}
The client requires the weight data to be retrieved remotely. We need to figure out two things: where do we want the data to be sent, and how do we want it to be sent.
This section investigates different options for transmitting data from a microprocessor. In a comparative performance study by Eridani et al., three protocols were compared: Bluetooth, Wi-Fi direct, and ESP-NOW ("a new protocol that allows multiple devices to communicate with each other without the use of Wi-Fi, with low power consumption" \cite{comparitiveEspnow}). 5 metrics were used in the tests: maximum range, transmission speed, latency, power usage, and signal resistance to obstructions \cite{comparitiveEspnow}. A brief summary of the performance of each protocol is shown in the below figure.
\begin{figure}[h]
	\centering
	\includegraphics[width=0.7\linewidth]{Figures/performance}
	\caption{Overall Performance of each Protocol}
	\label{fig:performance}
\end{figure}

According to Gaurav and Ninad, Bluetooth is far easier to setup than Wi-Fi. On the other hand, Wi-Fi has higher data transfer speed for high baud rates  \cite{comparativewifi}. In the context of capturing weight data of birds, there is not a large amount of data being captured; therefore, high data transfer speed is not needed. In light of this, Bluetooth was further investigated.

% ----------------------------------------------------
\ifstandalone
\bibliography{../Bibliography/References.bib}
\printnoidxglossary[type=\acronymtype,nonumberlist]
\fi
\end{document}
% ----------------------------------------------------