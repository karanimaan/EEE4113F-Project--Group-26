limiting%Power subsystem\documentclass[class=report,11pt,crop=false]{standalone}
% ----------------------------------------------------
% ----------------------------------------------------
\documentclass[class=report,11pt,crop=false]{standalone}
% Page geometry
\usepackage[a4paper,margin=20mm,top=25mm,bottom=25mm]{geometry}

% Font choice
\usepackage{lmodern}

\usepackage{lipsum}

% Use IEEE bibliography style
\bibliographystyle{IEEEtran}

% Line spacing
\usepackage{setspace}
\setstretch{1.20}

% Ensure UTF8 encoding
\usepackage[utf8]{inputenc}

% Language standard (not too important)
\usepackage[english]{babel}

% Skip a line in between paragraphs
\usepackage{parskip}

% For the creation of dummy text
\usepackage{blindtext}

% Math
\usepackage{amsmath}

% Header & Footer stuff
\usepackage{fancyhdr}
\pagestyle{fancy}
\fancyhead{}
\fancyhead[R]{\nouppercase{\rightmark}}
\fancyfoot{}
\fancyfoot[C]{\thepage}
\renewcommand{\headrulewidth}{0.0pt}
\renewcommand{\footrulewidth}{0.0pt}
\setlength{\headheight}{13.6pt}

% Epigraphs
\usepackage{epigraph}
\setlength\epigraphrule{0pt}
\setlength{\epigraphwidth}{0.65\textwidth}

% Colour
\usepackage{color}
\usepackage[usenames,dvipsnames]{xcolor}

% Hyperlinks & References
\usepackage{hyperref}
\definecolor{linkColour}{RGB}{77,71,179}
\hypersetup{
    colorlinks=true,
    linkcolor=linkColour,
    filecolor=linkColour,
    urlcolor=linkColour,
    citecolor=linkColour,
}
\urlstyle{same}

% Automatically correct front-side quotes
\usepackage[autostyle=false, style=ukenglish]{csquotes}
\MakeOuterQuote{"}

% Graphics
\usepackage{graphicx}
\graphicspath{{Images/}{../Images/}}
\usepackage{makecell}
\usepackage{transparent}

% SI units
\usepackage{siunitx}

% Microtype goodness
\usepackage{microtype}

% Listings
\usepackage[T1]{fontenc}
\usepackage{listings}
\usepackage[scaled=0.8]{DejaVuSansMono}

% Custom colours for listings
\definecolor{backgroundColour}{RGB}{250,250,250}
\definecolor{commentColour}{RGB}{73, 175, 102}
\definecolor{identifierColour}{RGB}{196, 19, 66}
\definecolor{stringColour}{RGB}{252, 156, 30}
\definecolor{keywordColour}{RGB}{50, 38, 224}
\definecolor{lineNumbersColour}{RGB}{127,127,127}
\lstset{
  language=Matlab,
  captionpos=b,
  aboveskip=15pt,belowskip=10pt,
  backgroundcolor=\color{backgroundColour},
  basicstyle=\ttfamily,%\footnotesize,        % the size of the fonts that are used for the code
  breakatwhitespace=false,         % sets if automatic breaks should only happen at whitespace
  breaklines=true,                 % sets automatic line breaking
  postbreak=\mbox{\textcolor{red}{$\hookrightarrow$}\space},
  commentstyle=\color{commentColour},    % comment style
  identifierstyle=\color{identifierColour},
  stringstyle=\color{stringColour},
   keywordstyle=\color{keywordColour},       % keyword style
  %escapeinside={\%*}{*)},          % if you want to add LaTeX within your code
  extendedchars=true,              % lets you use non-ASCII characters; for 8-bits encodings only, does not work with UTF-8
  frame=single,	                   % adds a frame around the code
  keepspaces=true,                 % keeps spaces in text, useful for keeping indentation of code (possibly needs columns=flexible)
  morekeywords={*,...},            % if you want to add more keywords to the set
  numbers=left,                    % where to put the line-numbers; possible values are (none, left, right)
  numbersep=5pt,                   % how far the line-numbers are from the code
  numberstyle=\tiny\color{lineNumbersColour}, % the style that is used for the line-numbers
  rulecolor=\color{black},         % if not set, the frame-color may be changed on line-breaks within not-black text (e.g. comments (green here))
  showspaces=false,                % show spaces everywhere adding particular underscores; it overrides 'showstringspaces'
  showstringspaces=false,          % underline spaces within strings only
  showtabs=false,                  % show tabs within strings adding particular underscores
  stepnumber=1,                    % the step between two line-numbers. If it's 1, each line will be numbered
  tabsize=2,	                   % sets default tabsize to 2 spaces
  %title=\lstname                   % show the filename of files included with \lstinputlisting; also try caption instead of title
}

% Caption stuff
\usepackage[hypcap=true, justification=centering]{caption}
\usepackage{subcaption}

% Glossary package
% \usepackage[acronym]{glossaries}
\usepackage{glossaries-extra}
\setabbreviationstyle[acronym]{long-short}

% For Proofs & Theorems
\usepackage{amsthm}

% Maths symbols
\usepackage{amssymb}
\usepackage{mathrsfs}
\usepackage{mathtools}

% For algorithms
\usepackage[]{algorithm2e}

% Spacing stuff
\setlength{\abovecaptionskip}{5pt plus 3pt minus 2pt}
\setlength{\belowcaptionskip}{5pt plus 3pt minus 2pt}
\setlength{\textfloatsep}{10pt plus 3pt minus 2pt}
\setlength{\intextsep}{15pt plus 3pt minus 2pt}

% For aligning footnotes at bottom of page, instead of hugging text
\usepackage[bottom]{footmisc}

% Add LoF, Bib, etc. to ToC
\usepackage[nottoc]{tocbibind}

% SI
\usepackage{siunitx}

% For removing some whitespace in Chapter headings etc
\usepackage{etoolbox}
\makeatletter
\patchcmd{\@makechapterhead}{\vspace*{50\p@}}{\vspace*{-10pt}}{}{}%
\patchcmd{\@makeschapterhead}{\vspace*{50\p@}}{\vspace*{-10pt}}{}{}%
\makeatother
\makenoidxglossaries

\newacronym{radar}{RADAR}{Radio Detection and Ranging}
	\begin{document}
	% ----------------------------------------------------
	\chapter{Power Subsystem (MBSLUN008)}
	\vspace{0.5cm}
	% ------------------------------------------
	\section{Introduction}
	The power subsystem is one of the most important subsystems in the design with respect to its functionality. This subsystem is responsible for powering up the micro-controller (MCU). It is very important for the design to have a reliable and safe power supply unit (PSU). This section outlines how the power supply unit design is divided into smaller sections, all with different responsibilities, requirements and specifications, but with the same goal: to achieve a reliable power supply source. The design uses a battery as its primary source of power. The subsection includes sections such as battery charging, overload protection, reverse polarity and a voltage regulation system. Designed were two circuits which work together to complete the whole PSU. First circuit is for battery charging, encapsulated with protection, then the voltage regulation circuit. As abovementioned, the PSU's goal is to provide a reliable power source to the micro-controller for more efficient performance. 

	\subsection{System overview}
	
	Shown below is the system overview of the whole power supply unit. 
	%insertimage of system
	
	\section{User requirements}
	\section{Specifications}
	
	\section{Design}
	\subsection{Battery charging}
	\vspace{0.5cm}
	Lithium-ion batteries are rechargeable,  The charging circuit should have a stable power supply. This supply can be achieved  by using a wall adapter, USB port, or another power source, such as a power bank. The charging circuit of the 18650 Li-ion batteries uses a BC745 PNP transistor as the driver of the circuit. The circuit diagram is shown below. The circuit takes in a voltage of between 8V - 5V. This circuit is designed to charge one battery at a time, making it more efficient and reliable. The charging circuit has two LED indicators. These LEDs are designed to indicate the level of the battery power when it is plugged to the charger. When the voltage is low, i.e the battery power is low, the RED LED will be much brighter than the GREEN LED. When the voltage now approaches the maximum voltage of 4.2V (actual voltage on a 3.7V Li-ion battery), the green LED dominates the RED and the user can know that the battery is charged. This is a sample and cost effective way to monitor the charging process of the batteries.  
	
	\vspace{0.5cm}
	
	\subsection{Battery Protection}
	\vspace{0.5cm}
	
	\subsubsection{Overload protection}
	\vspace{0.5cm}
	During the charging process, the 18650 battery must be protected against voltage and current overloading. The overload protection should keep note of key considerations and address to ensure both safety and optimal battery performance. The Li-ion battery has a nominal voltage of 3.7V, it is imperative to note that the specifications of the battery state that the charging voltage should not exceed 4.2V at 0.052A. A common approach involves incorporating a dedicated protection IC that monitors charging parameters such as voltage, current, and temperature. When an overload condition is detected, typically caused by excessive current flow or prolonged charging beyond safe limits, the protection IC triggers a mechanism to shutdown and disconnect the charging circuit from the battery. This prevents overcharging, which can lead to overheating, cell damage, or even fire hazards.
	
	In the circuit below, a 10k ohms variable resistor was used to vary the voltage of the circuit. Resistors to the values shown below are used to limit the current in the circuit. A power transistor LM395 is used in the overload protection as shown. LM395 is ideal for low power applications such as our small scale sensing design. These transistors act as high gain power transistors, and are capable of power, voltage, current limiting and heat/thermal overloading protection. It is very rare to find a device which is able to provide overload and thermal protection concurrently, which makes the LM395 very reliable to our design. The use of these transistors delivers simplicity and reliability to our design. The thermal limiting circuitry inside the LM395 will terminate the circuit should there be excessive amount of heat, protecting the battery from burning. 
	
	
	\subsubsection{Underload protection}
	\vspace{0.5cm}
	In designing undervoltage protection for a charging circuit intended for 18650 lithium-ion batteries, the primary objective is to prevent the battery from being over-discharged, which can lead to irreversible damage and compromise its lifespan. A typical implementation involves incorporating a dedicated undervoltage protection IC or circuitry that constantly monitors the battery's voltage level during discharge and charging cycles. When the battery voltage drops below a predefined threshold, indicating a critically low state of charge, the protection circuit triggers a disconnect mechanism to halt further discharge and protect the battery from damage. This safeguard ensures that the battery remains within safe operating limits, prolonging its lifespan and preventing potentially hazardous conditions. Additionally, integrated battery management systems (BMS) may include features such as automatic shutdown or alarm signaling to alert users when the battery reaches a low voltage condition, facilitating timely intervention and maintenance. By incorporating robust undervoltage protection mechanisms, the charging circuit can effectively preserve the health and integrity of 18650 lithium-ion batteries, promoting safe and reliable operation in various applications.
	
	
	\subsection{Reverse polarity protection}
	\vspace{0.5cm}
	The uHAT’s operation requires only a forward drop across all its components, if not
	protected from a reversed input this could burn, destroy, and saturates components. The
	input protection is modelled using a Power MOSFET connected to a Zener diode to
	maximise its efficiency. The power mosfet has a parasitic diode on its body which has a
	forward voltage drop of 1V, this is efficient because the voltage at the Drain (input) has to be
	always greater by at least volt to that of the source(output) in order for the circuit to conduct
	and allow current flow, furthermore a Zener diode ensures a regulated 1V at the source for
	current to flow from the gate to the source which would now mean that the voltage at the
	input has to be at least 2V greater than that of the source and this is how an reverse input
	polarity is achieved.

	\subsection{Voltage regulation}
	\section{Results and Testing}
	\section{Final design}
	\section{Conclusion and Recommendations}
	
\end{document}