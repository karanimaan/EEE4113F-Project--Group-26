% ----------------------------------------------------
% Housing Subsystem
% ----------------------------------------------------
\documentclass[class=report,11pt,crop=false]{standalone}
% Page geometry
\usepackage[a4paper,margin=20mm,top=25mm,bottom=25mm]{geometry}

% Font choice
\usepackage{lmodern}

\usepackage{lipsum}

% Use IEEE bibliography style
\bibliographystyle{IEEEtran}

% Line spacing
\usepackage{setspace}
\setstretch{1.20}

% Ensure UTF8 encoding
\usepackage[utf8]{inputenc}

% Language standard (not too important)
\usepackage[english]{babel}

% Skip a line in between paragraphs
\usepackage{parskip}

% For the creation of dummy text
\usepackage{blindtext}

% Math
\usepackage{amsmath}

% Header & Footer stuff
\usepackage{fancyhdr}
\pagestyle{fancy}
\fancyhead{}
\fancyhead[R]{\nouppercase{\rightmark}}
\fancyfoot{}
\fancyfoot[C]{\thepage}
\renewcommand{\headrulewidth}{0.0pt}
\renewcommand{\footrulewidth}{0.0pt}
\setlength{\headheight}{13.6pt}

% Epigraphs
\usepackage{epigraph}
\setlength\epigraphrule{0pt}
\setlength{\epigraphwidth}{0.65\textwidth}

% Colour
\usepackage{color}
\usepackage[usenames,dvipsnames]{xcolor}

% Hyperlinks & References
\usepackage{hyperref}
\definecolor{linkColour}{RGB}{77,71,179}
\hypersetup{
    colorlinks=true,
    linkcolor=linkColour,
    filecolor=linkColour,
    urlcolor=linkColour,
    citecolor=linkColour,
}
\urlstyle{same}

% Automatically correct front-side quotes
\usepackage[autostyle=false, style=ukenglish]{csquotes}
\MakeOuterQuote{"}

% Graphics
\usepackage{graphicx}
\graphicspath{{Images/}{../Images/}}
\usepackage{makecell}
\usepackage{transparent}

% SI units
\usepackage{siunitx}

% Microtype goodness
\usepackage{microtype}

% Listings
\usepackage[T1]{fontenc}
\usepackage{listings}
\usepackage[scaled=0.8]{DejaVuSansMono}

% Custom colours for listings
\definecolor{backgroundColour}{RGB}{250,250,250}
\definecolor{commentColour}{RGB}{73, 175, 102}
\definecolor{identifierColour}{RGB}{196, 19, 66}
\definecolor{stringColour}{RGB}{252, 156, 30}
\definecolor{keywordColour}{RGB}{50, 38, 224}
\definecolor{lineNumbersColour}{RGB}{127,127,127}
\lstset{
  language=Matlab,
  captionpos=b,
  aboveskip=15pt,belowskip=10pt,
  backgroundcolor=\color{backgroundColour},
  basicstyle=\ttfamily,%\footnotesize,        % the size of the fonts that are used for the code
  breakatwhitespace=false,         % sets if automatic breaks should only happen at whitespace
  breaklines=true,                 % sets automatic line breaking
  postbreak=\mbox{\textcolor{red}{$\hookrightarrow$}\space},
  commentstyle=\color{commentColour},    % comment style
  identifierstyle=\color{identifierColour},
  stringstyle=\color{stringColour},
   keywordstyle=\color{keywordColour},       % keyword style
  %escapeinside={\%*}{*)},          % if you want to add LaTeX within your code
  extendedchars=true,              % lets you use non-ASCII characters; for 8-bits encodings only, does not work with UTF-8
  frame=single,	                   % adds a frame around the code
  keepspaces=true,                 % keeps spaces in text, useful for keeping indentation of code (possibly needs columns=flexible)
  morekeywords={*,...},            % if you want to add more keywords to the set
  numbers=left,                    % where to put the line-numbers; possible values are (none, left, right)
  numbersep=5pt,                   % how far the line-numbers are from the code
  numberstyle=\tiny\color{lineNumbersColour}, % the style that is used for the line-numbers
  rulecolor=\color{black},         % if not set, the frame-color may be changed on line-breaks within not-black text (e.g. comments (green here))
  showspaces=false,                % show spaces everywhere adding particular underscores; it overrides 'showstringspaces'
  showstringspaces=false,          % underline spaces within strings only
  showtabs=false,                  % show tabs within strings adding particular underscores
  stepnumber=1,                    % the step between two line-numbers. If it's 1, each line will be numbered
  tabsize=2,	                   % sets default tabsize to 2 spaces
  %title=\lstname                   % show the filename of files included with \lstinputlisting; also try caption instead of title
}

% Caption stuff
\usepackage[hypcap=true, justification=centering]{caption}
\usepackage{subcaption}

% Glossary package
% \usepackage[acronym]{glossaries}
\usepackage{glossaries-extra}
\setabbreviationstyle[acronym]{long-short}

% For Proofs & Theorems
\usepackage{amsthm}

% Maths symbols
\usepackage{amssymb}
\usepackage{mathrsfs}
\usepackage{mathtools}

% For algorithms
\usepackage[]{algorithm2e}

% Spacing stuff
\setlength{\abovecaptionskip}{5pt plus 3pt minus 2pt}
\setlength{\belowcaptionskip}{5pt plus 3pt minus 2pt}
\setlength{\textfloatsep}{10pt plus 3pt minus 2pt}
\setlength{\intextsep}{15pt plus 3pt minus 2pt}

% For aligning footnotes at bottom of page, instead of hugging text
\usepackage[bottom]{footmisc}

% Add LoF, Bib, etc. to ToC
\usepackage[nottoc]{tocbibind}

% SI
\usepackage{siunitx}

% For removing some whitespace in Chapter headings etc
\usepackage{etoolbox}
\makeatletter
\patchcmd{\@makechapterhead}{\vspace*{50\p@}}{\vspace*{-10pt}}{}{}%
\patchcmd{\@makeschapterhead}{\vspace*{50\p@}}{\vspace*{-10pt}}{}{}%
\makeatother
\makenoidxglossaries

\newacronym{radar}{RADAR}{Radio Detection and Ranging}
\begin{document}
	% ----------------------------------------------------
	\chapter{Housing Subsystem (VSTMIC004)}
	\vspace{0.5cm}
	% ---------------------------------------------------
	\section{Introduction}
	The aim of this subsection is to discuss the design and implementation of the housing of the final design. This housing will physically contain and protect all other subsystems as well as facilitate their operation.
	\section{Requirement Analysis}
	
	\begin{table}[h!]
		\centering
		\caption{Functional Specifications of the Housing Subsystem}
		\label{tab:S1}
		\begin{tabularx}{0.8\textwidth}{ 
				| >{\centering\arraybackslash}m 
				| >{\centering\arraybackslash}b 
				| >{\centering\arraybackslash}s |}
			\hline
			\textbf{User   Requirement} & \textbf{Specification   Description}                                     & \textbf{Specification   no.} \\ \hline
			Portable                    & The final internal size of the design must be 100x100x50mm 			   & HS1                          \\ \hline
			Weight Capacity        & The final design must be able to support the weight of all other subsystems including the maximum weight of the scale. & HS2 \\ \hline
			Weight Distribution & The final design must be able to contain all other subsystems in such a way that the overall center of gravity of the final design sits in the middle of the housing subsystem & HS3 \\ \hline
			External Push Button & The housing must support an external push button that can directly interface with the tare function developed in the Sensing Subsection & HS4 \\ \hline
		\end{tabularx}
	\end{table}

\section{Design Process}
\subsection{Housing Sections}
The sections that will make up the housing subsystem can be broadly categorized into two sections; the outer section and the inner section. The outer section will directly interface with the users while the inner section will interface and house the other subsections. 

The outer section will need to comply not only with all the user requirements but will also need to comply with any further material considerations, as will be discussed in section [next]. The outer section will directly interact with specifications HS1, HS2, HS3 and HS4.

The internal section will primarily govern the HS3 specification as it will determine how the other subsections will be contained within the final housing. It will need to distribute the weight of all other subsections such that the accuracy of the sensing subsection is not affected.

\subsection{Material Requirements}
As emphasized during the literature review, material choice is a component that is of particular importance regarding this subsection. Not only is material choice a variable in the execution of the user requirements, but it is important to choose materials that won’t harm or pose any danger to the wildlife that will be interacting with the final design. From the research conducted in the literature review, the following material specifications have been designated,

	\begin{table}[h!]
	\centering
	\caption{Material Requirements}
	\label{tab:H1}
	\begin{tabularx}{0.8\textwidth}{ 
			| >{\centering\arraybackslash}m 
			| >{\centering\arraybackslash}b 
			| >{\centering\arraybackslash}s |}
		\hline
		\textbf{Material   Requirement} & \textbf{Requirement   Description}                                     & \textbf{Requirement   no.} \\ \hline
		Avian-safe                   & The materials used should not be toxic in any way towards avian species. The materials used should not have any sharp edges or corners that can harm avians that will be interacting with the final design.			   & HSM1 \\ \hline
		User-safe        & The materials used should not be toxic in any to the users that will interact with the final design. The materials used should not have any sharp edges or corners that can harm users that will interact with the final design. & HSM2 \\ \hline
		Density Consistency & The materials chosen should be evenly dense such that the center of gravity of the final housing design lies equidistant between all edges of the outer section & HSM3 \\ \hline
		Ease of Modularity & The materials chosen must be easy to modify (either permanent or temporary modifications) by both the designers of the subsection and end users. & HSM4 \\ \hline
		Environmental Resistance & The materials chosen should be able to resist damage from water and sunlight such that, even if prolonged for long periods of time, the functionality of every subsection (including this one) remains unaffected. & HSM5 \\ \hline
		Budget Constraint & The total cost of all the materials require should fall within the budget for the entire subsection (R450) & HSM6 \\ \hline 
	\end{tabularx}
\end{table}

\subsection{Material Options}
Taking into account all viable materials available at local hardware stores, the following list has been compiled alongside all material requirements they fulfill. The following list includes the associated product code at BuCo as of 2024/05/12. From this list, materials that fulfill all requirements will be analyzed further based on their cost-effectiveness.

	\begin{table}[h!]
	\centering
	\caption{Material Requirements}
	\label{tab:H2}
	\begin{tabularx}{0.8\textwidth}{ 
			| >{\centering\arraybackslash}m 
			| >{\centering\arraybackslash}b
			| >{\centering\arraybackslash}x 
			| >{\centering\arraybackslash}s |}
		\hline
		\textbf{Product Type} & \textbf{BuCo SKU}  & \textbf{Requirements Fulfilled} & \textbf{Requirement Discussion} \\ \hline
		Plywood & 1274706 & HSM1,HSM2,HSM3,HSM4, HSM5, HSM6 & Fulfills all material requirements \\ \hline
		Hardboard & 1292581 & HSM1, HSM2, HSM3, HSM4, HSM6 & Does not fulfill the HSM5 requirement as it has little moisture resistance. \\ \hline
		Chipboard & 1283947 & HSM1, HSM2, HSM3, HSM6 & Does not fulfill the HSM4 requirement as it cannot be altered easily without the use of adhesives, which will compromise the structural integrity of the housing design. It also does not meet the HSM5 requirement due to its tendency to chip and shatter over time. \\ \hline
		SupaWood & 1276088 & HSM1,HSM2,HSM3,HSM4, HSM5, HSM6 & Fulfills all material requirements. \\ \hline
		Laminated Pine Shelving & 1124588 & HSM1,HSM2,HSM3,HSM4, HSM5 & Does not fulfill the HSM6 requirement due to it being outside of the subsystem budget. \\ \hline
	\end{tabularx}`
\end{table}

From the above list the two most viable materials, including their sizing and pricing, are cataloged below,

	\begin{tabularx}{0.8\textwidth}{ 
		\caption{Furthered Material Analysis}
		\label{tab:H3}
		| >{\centering\arraybackslash}m 
		| >{\centering\arraybackslash}b 
		| >{\centering\arraybackslash}s |}
	\hline
	\textbf{Product Type} & \textbf{Dimensions (mm)}  & \textbf{Pricing} \\ \hline
	Plywood & 2440x1220x3.6 & R426.98 \\ \hline
	SupaWood & 2750x1830x3 & R302. 02 \\ \hline 

\end{tabularx}
\end{table}



% ----------------------------------------------------
	\ifstandalone
	\bibliography{../Bibliography/References.bib}
	\printnoidxglossary[type=\acronymtype,nonumberlist]
	\fi
	
\end{document}


