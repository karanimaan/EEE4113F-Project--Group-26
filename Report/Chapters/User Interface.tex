% ----------------------------------------------------
% Recommendations
% ----------------------------------------------------
\documentclass[class=report,11pt,crop=false]{standalone}
% Page geometry
\usepackage[a4paper,margin=20mm,top=25mm,bottom=25mm]{geometry}

% Font choice
\usepackage{lmodern}

\usepackage{lipsum}

% Use IEEE bibliography style
\bibliographystyle{IEEEtran}

% Line spacing
\usepackage{setspace}
\setstretch{1.20}

% Ensure UTF8 encoding
\usepackage[utf8]{inputenc}

% Language standard (not too important)
\usepackage[english]{babel}

% Skip a line in between paragraphs
\usepackage{parskip}

% For the creation of dummy text
\usepackage{blindtext}

% Math
\usepackage{amsmath}

% Header & Footer stuff
\usepackage{fancyhdr}
\pagestyle{fancy}
\fancyhead{}
\fancyhead[R]{\nouppercase{\rightmark}}
\fancyfoot{}
\fancyfoot[C]{\thepage}
\renewcommand{\headrulewidth}{0.0pt}
\renewcommand{\footrulewidth}{0.0pt}
\setlength{\headheight}{13.6pt}

% Epigraphs
\usepackage{epigraph}
\setlength\epigraphrule{0pt}
\setlength{\epigraphwidth}{0.65\textwidth}

% Colour
\usepackage{color}
\usepackage[usenames,dvipsnames]{xcolor}

% Hyperlinks & References
\usepackage{hyperref}
\definecolor{linkColour}{RGB}{77,71,179}
\hypersetup{
    colorlinks=true,
    linkcolor=linkColour,
    filecolor=linkColour,
    urlcolor=linkColour,
    citecolor=linkColour,
}
\urlstyle{same}

% Automatically correct front-side quotes
\usepackage[autostyle=false, style=ukenglish]{csquotes}
\MakeOuterQuote{"}

% Graphics
\usepackage{graphicx}
\graphicspath{{Images/}{../Images/}}
\usepackage{makecell}
\usepackage{transparent}

% SI units
\usepackage{siunitx}

% Microtype goodness
\usepackage{microtype}

% Listings
\usepackage[T1]{fontenc}
\usepackage{listings}
\usepackage[scaled=0.8]{DejaVuSansMono}

% Custom colours for listings
\definecolor{backgroundColour}{RGB}{250,250,250}
\definecolor{commentColour}{RGB}{73, 175, 102}
\definecolor{identifierColour}{RGB}{196, 19, 66}
\definecolor{stringColour}{RGB}{252, 156, 30}
\definecolor{keywordColour}{RGB}{50, 38, 224}
\definecolor{lineNumbersColour}{RGB}{127,127,127}
\lstset{
  language=Matlab,
  captionpos=b,
  aboveskip=15pt,belowskip=10pt,
  backgroundcolor=\color{backgroundColour},
  basicstyle=\ttfamily,%\footnotesize,        % the size of the fonts that are used for the code
  breakatwhitespace=false,         % sets if automatic breaks should only happen at whitespace
  breaklines=true,                 % sets automatic line breaking
  postbreak=\mbox{\textcolor{red}{$\hookrightarrow$}\space},
  commentstyle=\color{commentColour},    % comment style
  identifierstyle=\color{identifierColour},
  stringstyle=\color{stringColour},
   keywordstyle=\color{keywordColour},       % keyword style
  %escapeinside={\%*}{*)},          % if you want to add LaTeX within your code
  extendedchars=true,              % lets you use non-ASCII characters; for 8-bits encodings only, does not work with UTF-8
  frame=single,	                   % adds a frame around the code
  keepspaces=true,                 % keeps spaces in text, useful for keeping indentation of code (possibly needs columns=flexible)
  morekeywords={*,...},            % if you want to add more keywords to the set
  numbers=left,                    % where to put the line-numbers; possible values are (none, left, right)
  numbersep=5pt,                   % how far the line-numbers are from the code
  numberstyle=\tiny\color{lineNumbersColour}, % the style that is used for the line-numbers
  rulecolor=\color{black},         % if not set, the frame-color may be changed on line-breaks within not-black text (e.g. comments (green here))
  showspaces=false,                % show spaces everywhere adding particular underscores; it overrides 'showstringspaces'
  showstringspaces=false,          % underline spaces within strings only
  showtabs=false,                  % show tabs within strings adding particular underscores
  stepnumber=1,                    % the step between two line-numbers. If it's 1, each line will be numbered
  tabsize=2,	                   % sets default tabsize to 2 spaces
  %title=\lstname                   % show the filename of files included with \lstinputlisting; also try caption instead of title
}

% Caption stuff
\usepackage[hypcap=true, justification=centering]{caption}
\usepackage{subcaption}

% Glossary package
% \usepackage[acronym]{glossaries}
\usepackage{glossaries-extra}
\setabbreviationstyle[acronym]{long-short}

% For Proofs & Theorems
\usepackage{amsthm}

% Maths symbols
\usepackage{amssymb}
\usepackage{mathrsfs}
\usepackage{mathtools}

% For algorithms
\usepackage[]{algorithm2e}

% Spacing stuff
\setlength{\abovecaptionskip}{5pt plus 3pt minus 2pt}
\setlength{\belowcaptionskip}{5pt plus 3pt minus 2pt}
\setlength{\textfloatsep}{10pt plus 3pt minus 2pt}
\setlength{\intextsep}{15pt plus 3pt minus 2pt}

% For aligning footnotes at bottom of page, instead of hugging text
\usepackage[bottom]{footmisc}

% Add LoF, Bib, etc. to ToC
\usepackage[nottoc]{tocbibind}

% SI
\usepackage{siunitx}

% For removing some whitespace in Chapter headings etc
\usepackage{etoolbox}
\makeatletter
\patchcmd{\@makechapterhead}{\vspace*{50\p@}}{\vspace*{-10pt}}{}{}%
\patchcmd{\@makeschapterhead}{\vspace*{50\p@}}{\vspace*{-10pt}}{}{}%
\makeatother
\makenoidxglossaries

\newacronym{radar}{RADAR}{Radio Detection and Ranging}

\begin{document}
	% ----------------------------------------------------
	\chapter{User Interface \label{ch:user-interface}}
	
	\vspace{0.5cm}
	% ----------------------------------------------------
	
	\section{Introduction}
	This chapter deals with the user interface of the system. The user interface is how the client will be able to make use of the system.
	
	\section{Requirements}
	
		\subsection{User Requirements}
		\begin{enumerate}
			\item Display weight data on app.
			\item App must have interface for manual data.
			\item Other data should be automated (time, location).
			\item Data must be saved to a spreadsheet. It should have the following fields: FID, Date, Time, ID, Ring type, Sex, Mass, Day status, Day status, Day, Weekday, Term, Name, Weight Session, Location, Notes, People Count 1, People Count 2.
			
		\end{enumerate}
		
		\subsection{Functional Requirements}
		\begin{enumerate}
			\item Establish connection between smart scale (Arduino) and phone (via Bluetooth or Wi-Fi) 
			\item The GUI should have a text box to input the bird's ID.
			\item A CSV file needs to be created where the data will be written.
		\end{enumerate}
			
			
		
	
		
	
	\section{Design Process}
		
		\begin{figure}[h!]
			\centering
			\includegraphics[width=0.9\linewidth]{Figures/Starling data block diagram.drawio.pdf}
			\caption{Block diagram of subsystem}
		\end{figure}
		
		This subsystem exists in two environments: the Arduino, and the mobile device. The Arduino needs to be programmed to send the weight readings to the mobile device. A mobile app needs to be developed to receive the weight readings. The app needs to provide a GUI that allows the user to input the bird's ID. The app should save the data to a spreadsheet for the user to view.
		
		\subsection{Sending readings}
		
		%\subsection{Receiving readings}
		
		\subsection{Graphical User Interface}
		The Graphical User Interface (GUI) is responsible for displaying the live reading coming from the Arduino, and allowing the user to capture that reading and input the bird ID. The GUI also needs a save button to save the data to a file, so that the user can view the data in a spreadsheet.
		The prototype of the GUI is shown below in figure \ref{fig:gui-prototype}.
				
		\begin{figure}[h!]
			\centering
			\includegraphics[scale=1]{"Figures/GUI prototype"}
			\caption{GUI prototype}
			\label{fig:gui-prototype}
		\end{figure}
		
		\subsection{Saving data}
		
		
		The spreadsheet should have the following fields: FID, Date, Time, ID, Ring type, Sex, Mass, Day status, Day status, Day (of week), Weekday, Term, Name, Weighing Session, Location, Notes, People Count 1, People Count 2.
		
		FID - pk
		Date, Time, Day, Weekday, Weighing Session, Location - Kotlin
		ID, Day status, Name, Notes, People Count 1, People Count 2 - user
		Ring type, Sex - BirdID table
		Mass - Arduino
		Term - Terms table
		
	
	\section{Implementation}
		\subsection{Mobile app}
		The mobile app was developed using Android Studio IDE was used with the Kotlin programming language. The app is targeted for Android devices. 
	
		
	
	%\section{Final Design}
	
	%\section{Testing}
	\section{Acceptance Test Procedure}
	Testing was done on a Samsung S9 phone running Android 10.
	\subsection{Unit Testing}

	
	\subsection{User Acceptance Testing}
	
	\section{Conclusion}
	
	% ----------------------------------------------------
	\ifstandalone
	\bibliography{../Bibliography/References.bib}
	\printnoidxglossary[type=\acronymtype,nonumberlist]
	\fi
\end{document}
% ----------------------------------------------------