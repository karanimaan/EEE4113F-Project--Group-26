% ----------------------------------------------------
% Recommendations
% ----------------------------------------------------
\documentclass[class=report,11pt,crop=false]{standalone}
\input{../Style/ChapterStyle.tex}
\input{../FrontMatter/Glossary.tex}

\begin{document}
	% ----------------------------------------------------
	\chapter{User Interface \label{ch:user-interface}}
	
	\vspace{0.5cm}
	% ----------------------------------------------------
	
	\section{Introduction}
	This chapter deals with the user interface of the system. The user interface is how the client will be able to make use of the system.
	
	\section{Requirements}
	
		\subsection{User Requirements}
		\begin{enumerate}
			\item Display weight data on app.
			\item App must have interface for manual data.
			\item Other data should be automated (time, location).
			\item Data must be saved to a spreadsheet. It should have the following fields: FID, Date, Time, ID, Ring type, Sex, Mass, Day status, Day status, Day, Weekday, Term, Name, Weight Session, Location, Notes, People Count 1, People Count 2.
			
		\end{enumerate}
		
		\subsection{Functional Requirements}
		\begin{enumerate}
			\item Establish connection between smart scale (Arduino) and phone (via Bluetooth or Wi-Fi) 
			\item The GUI should have a text box to input the bird's ID.
			\item A CSV file needs to be created where the data will be written.
		\end{enumerate}
			
			
		
	
		
	
	\section{Design Process}
		
		\begin{figure}[h!]
			\centering
			\includegraphics[width=0.9\linewidth]{Figures/Starling data block diagram.pdf}
			\caption{Block diagram of subsystem}
		\end{figure}
		
		This subsystem exists in two environments: the Arduino, and the mobile device. The Arduino needs to be programmed to send the weight readings to the mobile device. A mobile app needs to be developed to receive the weight readings. The app needs to provide a GUI that allows the user to input the bird's ID. The app should save the data to a spreadsheet for the user to view.
		
		\subsection{Sending readings}
		There are several technologies to transmit data such as: Wi-Fi, Bluetooth, and ESP-NOW. 
		As discussed in the Literature Review, Bluetooth is ideal because it is the power efficient and its speed is sufficient because only several bytes of data (for the weight) are being transmitted.
		It was also discussed that internet has the advantage of ease of use for the user. However, internet access is not feasible as weighing sessions span outside of \textit{eduroam} coverage.
		There are two types of Bluetooth: Bluetooth Classic and Bluetooth Low Energy (BLE). BLE is a newer technology that is even more power efficient than Bluetooth Classic; therefore, Bluetooth Low Energy was chosen.
		
		BLE works differently to Bluetooth Classic. Bluetooth Classic is based on an asynchronous serial connection. BLE, on the other hand, works like a community bulletin board \cite{ble}. The peripheral device posts data for central devices to read \cite{ble}.This model is shown below in figure \ref{fig:ble-bulletin-board-model}.
		
		\begin{figure}[h!]
			\centering
			\includegraphics[scale=0.7]{Figures/ble-bulletin-board-model}
			\caption{BLE Bulletin Board Model \cite{ble}}
			\label{fig:ble-bulletin-board-model}
		\end{figure}
		
		%\subsection{Receiving readings}
		
		\subsection{Graphical User Interface}
		The Graphical User Interface (GUI) is responsible for displaying the live reading coming from the Arduino, and allowing the user to capture that reading and input the bird ID. The GUI also needs a save button to save the data to a file, so that the user can view the data in a spreadsheet.
		The prototype of the GUI is shown below in figure \ref{fig:gui-prototype}.
				
		\begin{figure}[h!]
			\centering
			\includegraphics[scale=1]{"Figures/GUI prototype"}
			\caption{GUI prototype}
			\label{fig:gui-prototype}
		\end{figure}
		
		\subsection{Saving data}
		
		
		The spreadsheet should have the following fields: FID, Date, Time, ID, Ring type, Sex, Mass, Day status, Day status, Day (of week), Weekday, Term, Name, Weighing Session, Location, Notes, People Count 1, People Count 2.
		
		FID - pk
		Date, Time, Day, Weekday, Weighing Session, Location - Kotlin
		ID, Day status, Name, Notes, People Count 1, People Count 2 - user
		Ring type, Sex - BirdID table
		Mass - Arduino
		Term - Terms table
		
	
	\section{Implementation}
	
		\subsection{Arduino}
		A program was developed and uploaded to the Arduino. This program, \href{https://github.com/karanimaan/EEE4113F-Project--Group-26/blob/main/send_data/send_data.ino}{send\_data.ino}, can be found under the \textit{send\_data} folder in the project repository.
	
		\subsection{Mobile app}
		The mobile app was developed using Android Studio IDE was used with the Kotlin programming language. The app is targeted for Android devices. 
	
		
	
	%\section{Final Design}
	
	%\section{Testing}
	\section{Acceptance Test Procedure}
	Testing was done on a Samsung S9 phone running Android 10.
	\subsection{Unit Testing}

	
	\subsection{User Acceptance Testing}
	
	\section{Conclusion}
	
	% ----------------------------------------------------
	\ifstandalone
	\bibliography{../Bibliography/References.bib}
	\printnoidxglossary[type=\acronymtype,nonumberlist]
	\fi
\end{document}
% ----------------------------------------------------